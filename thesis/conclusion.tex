
\lhead[\chaptername~\thechapter]{\rightmark}

\rhead[\leftmark]{}

\lfoot[\thepage]{}

\cfoot{}

\rfoot[]{\thepage}

\chapter{Conclusion}
\label{conclusion}

% \begin{paragraph}{Using object re-identification}
% The problem of re-identification \cite{zheng2016person} is an emerging topic in recent years. Although most of the work only deals with person or pedestrians, the idea of re-identification is very suitable for our task of discovering objects of joint attention. During our experiment, we observed that during one specific task, there are certain objects that will attract joint attention multiple times. By adopting the idea of object re-identification, we may be able to find the most relevant object of joint attention, which may be of use for future studies on psychology and group behaviour.
% \end{paragraph}

\section{Summary}

...

\section{Future Work}

...

\subsection*{Image Classification}

...

\subsection*{Image Retrieval}

...

