
\lhead[\chaptername~\thechapter]{\rightmark}

\rhead[\leftmark]{}

\lfoot[\thepage]{}

\cfoot{}

\rfoot[]{\thepage}

\chapter{Conclusion}
\label{conclusion}

% \begin{paragraph}{Using object re-identification}
% The problem of re-identification \cite{zheng2016person} is an emerging topic in recent years. Although most of the work only deals with person or pedestrians, the idea of re-identification is very suitable for our task of discovering objects of joint attention. During our experiment, we observed that during one specific task, there are certain objects that will attract joint attention multiple times. By adopting the idea of object re-identification, we may be able to find the most relevant object of joint attention, which may be of use for future studies on psychology and group behaviour.
% \end{paragraph}

\section{Summary}

In this thesis, it is demonstrated that efficient models can be obtained by taking advantage of the requirements of each user. 
We applied our idea to both image classification and image retrieval tasks.
In image classification, hierarchical CNN architecture is used to learn the multi-class classification. 
Then, a personalized model, a few branches of the hierarchical CNN, is distributed to each user according to their requirements.
Experiments on CIFAR-10 showed that average users could use a smaller, memory efficient model while maintaining the classification accuracy.
For image retrieval task, our idea is used to reduce the area of the search space to increase the search speed for each user.
By obtaining user requirements in the form of classification labels, each subspace is inspected whether they contain images with any of the user-required labels. 
By finding the relevant sub-spaces and performing the search only on them, a personalized area of search is determined for each user. 
Experimental results revealed that our method works as expected on artificial data whereas it still needs more work to be done for the real-life scenario.

\section{Future Work}

Possible future experiments and the direction of our research will be discussed in the following sections.

\subsection*{Image Classification}

According to our experimental results, each user can obtain a personal, memory-efficient CNN model by using hierarchical CNN.
Although these results are very promising, CIFAR-10 is a relatively small dataset and maintaining the accuracy of the models is not a hard task.
Therefore, the hierarchical CNN method should be tested in a larger dataset to see its true potential. 
However, there are various things to consider before experimenting on a big dataset.

The hierarchy of our model is following a binary tree structure. 
However, there are so many possibilities to construct the hierarchy. 
One example would be a flexible structure, where the branching operation splits the model into 2 or more branches, instead of only 2. 
Even though current experimental results suggest that the accuracy-memory trade-off is better when the depth of the tree is larger, an accuracy drop is expected when the hierarchical model is tested on a bigger dataset.
Furthermore, there is a need to examine alternative hierarchical clustering methods, such as other top-down and bottom-up approaches, to improve memory reduction according to user-required class labels.

The most important future task would be to test our model with real-life user data as user-required classification labels are generated artificially.
One approach of obtaining the real user data would be like the following scheme.
Experiment user group is divided into training and test users. 
We collect egocentric videos of daily activities from the training users.
A set of classification labels are obtained by classifying each user's videos.
These sets of class labels are used to hierarchically cluster the labels to obtain our hierarchy. 
Afterwards, hierarchical CNN with the obtained hierarchy is trained. 
Then, test users are given the whole hierarchical CNN and they start to use it according to our scenario. 
After a fixed time of using the whole model, each user's model is optimized according to the previously classified labels.
The users in the experiment group should be from diverse backgrounds to cover the whole classification labels. 
In this way, an accurate hierarchy can be obtained.


\subsection*{Image Retrieval}

In our experiments, IndexIVFPQ (Inverted File Index + Product Quantization) implementation of FAISS is used as our baseline. 
Current state-of-the-art implementation is IndexIVFPQ used with HNSW (Hierarchical Navigable Small World). 
For our future experiment, we will implement our idea on top of HNSW+IVFPQ to speed up the search according to our idea of reducing the search space according to user requirements.

Experimental results revealed that the results with real-life data are not as good as we expected. 
Possible reasons are discussed in the previous chapter. 
This time, we used videos from the daily-life of only one person for 2 days, first day for training and the other day for testing. 
In the future, longer videos from many users will be used in our experiments.
In order to eliminate the bad training and test frames, a better sampling strategy should be employed.

Places365, a scene recognition dataset, is used in our experiments. 
Because of the existence of many correlated class labels, the accuracy of classifying the labels is relatively low, therefore correctly finding the user-required label is difficult. 
In our future experiments, other datasets can be used to examine our method further. 

We will extend our experiments further to object classification. 
An example scenario will be as the following.
A large object classification dataset, such as ImageNet, will be used to build our search database.
In the training phase of our method, objects seen by the users will be collected to form user-required class labels for each user. 
Afterwards, test users will capture object images from their daily life to search for the most similar objects. 
We hope to examine our method's strengths even more with the object classification task.


